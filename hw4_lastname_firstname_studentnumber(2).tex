\documentclass[a4paper]{article}

\usepackage{sectsty}
\usepackage{graphicx}
\usepackage{listings}
\usepackage{subcaption}
\usepackage{framed}
\usepackage[]{enumitem}
\sectionfont{\fontsize{12}{15}\selectfont}
\usepackage[left=90.00pt, right=90.00pt, top=100.00pt, bottom=100.00pt]{geometry}
\title{Homework Assigment 4: Modelling pandemics \\ \large Scientific Software / Technisch Wetenschappelijke Software 2020}
\author{If I do not change this, the person grading my report will not know who I am!\\ It could be a good idea to change this before writing my report.}

\lstset{
	language=c++,
	%    basicstyle={\ttfamily \small},
	basicstyle={\ttfamily \small},
	%    keywordstyle=\underline,
	numberstyle={\footnotesize},
	%    morekeywords={ones,mod,isprime,inline,unique,factor,@},
	%    flexiblecolumns=false,
	%    emph={gamma,beta},
	%    emphstyle=,
	columns=fullflexible,
	%    columns=flexible,
	%    commentstyle={\slshape},
	%    commentstyle={\normalfont},
	commentstyle={\ttfamily},
	stringstyle={\ttfamily \bfseries},
	showstringspaces=false,
	%    indent=1em,
	%    xleftmargin=0.5em,
	breaklines=false,
	%    frame={l},
	captionpos={t},
	upquote=true, % such that we can copy-paste the code...
	%    mathescape=true,
	%    frame=single,     % boxed in a single line
	%    frame=L,          % double line on the left
	%    frame=l,          % single line on the left
}
\newcommand{\answer}[1]{\vspace{-0.75em}\begin{framed} #1 \end{framed}\vspace{-0.75em}}
\begin{document}

\maketitle
\noindent \textbf{Number of hours spent:} REPLACE WITH THE CORRECT VALUE
\section*{Practical information (Remove this section before submitting your re-
port)}
\textbf{Do not change the margins of this document!}\\


Here we show how to include C++-code and a figure:
\begin{lstlisting}
#include <iostream>
#include "vector.hpp"

int main(int argc, char const *argv[])
{
	tws::vector<double> x(3,1.);
	std::cout<<x<<std::endl;
	return 0;
}   
\end{lstlisting}

\begin{figure}[!h]
	\centering
	%\includegraphics[width=\textwidth]{}
	\caption{A description of my figure}
\end{figure}
\section*{General}
\begin{enumerate}[label={(G.\arabic*)}]
	\item Which important concepts, \verb|C++|-specific syntax, or tools from the lectures and exercise
sessions did you use? Where? Mentioning four concepts suffices. For each concept, limit
your discussion to 3 lines. You can forward reference to your answers for the following
questions.
	\answer{}
	\item \textbf{Optional:} Are there changes or improvements you wanted to make but where unable to
implement due to lack of time? If yes, briefly describe these.
	\answer{}
	\item \textbf{Optional:} Are there other general comments you want to make?
	\answer{}
\end{enumerate}
\section*{Part I: Simulating the pandemic}
\begin{enumerate}[label={(I.\arabic*)}]
	\item \textbf{Optional:} Have you made any modification to your IVP solvers based on the feedback of the second homework?
	\answer{}
	\item Your IVP solvers must now accept more general ordinary differential equations. Which
C++ features did you need to use to achieve this extra flexibility? What are the advantages
of this approach? (max. 3 lines)
\answer{}
\item For \verb|simulation2.cpp| it is asked how can you pass the differential
equation to your IVP solver? Discuss all possibilities. (max. 10 lines)
	\answer{}
\item How would you write an IVP solver in Fortran that accepts more general differential
equations? (max. 1 line)
\answer{Name of the sought for implementation pattern.}
	\item Compare your Fortran code and your \verb|C++| code.
	\begin{enumerate}
	\item What \verb|C++| specific features did you use? (Keep it short! A list of features suffices.
max. 2 lines)
	\answer{\hspace{-1cm} Feature 1, Feature 2, ..}
	\item Are there features that you used in Fortran but that are not available in \verb|C++|? (Keep
it short! A list of features suffices. max. 2 lines)
	\answer{}
	\item Are there design decisions that you have to take into account when working with \verb|C++|
that you do not have to make (or can not make) in Fortran? (max. 6 lines)
\answer{}
	\item \textbf{Optional:} Was it easier to implement the functionality in Fortran or in \verb|C++|?
	\answer{}
	\end{enumerate}

	\item In \verb|simulation2.cpp|, how did you avoid using an explicit for or while loop when evaluating the right-hand side of the differential equation? (max. 2 lines)
	\answer{\texttt{function from the stl}: some more information on how this function works}
	\item In \verb|simulation2.cpp|, how did you avoid using an explicit for or while loop when filling
the vector with the initial values? (max. 2 lines)
	\answer{\texttt{function from the stl}: some more information on how this function works}
	\item \textbf{Optional}: Are there other design decisions you want to mention?
	\answer{}
\end{enumerate}
\subsection*{Verify your code for the SIQRD model}
Do not forget to also include your figures separately in your ZIP!!!! (remove this line in your
submission)
\begin{figure}[!h]
	\centering
	\begin{subfigure}{0.32\linewidth}
		\centering
		%\includegraphics[width=\textwidth]{}
	\end{subfigure}
	\begin{subfigure}{0.32\linewidth}
		\centering
		%\includegraphics[width=\textwidth]{}
	\end{subfigure}
	\begin{subfigure}{0.32\linewidth}
		\centering
		%\includegraphics[width=\textwidth]{}
	\end{subfigure}
\end{figure}
\subsection*{Verify your code for the general ODE}
\answer{How did you verify your code for the general ODE?}
\section*{Part II: Parameter estimation from observations}
\begin{enumerate}[label={(II.\arabic*)}]
	\item \textbf{How did you optimize the performance (execution time and memory usage) of
the parameter estimator code?}
\begin{enumerate}
\item What part of the code has the most influence on the execution time (and therefore
needs to be implemented as efficient as possible)? (max. 1 line)
\answer{}
\item How did you avoid unnecessary memory usage or copies? (max. 4 lines)
\answer{}
\item What tools did you use to assess the performance of your code? (max. 4 lines)
\answer{}
\item Are there other things you have done to improve performance?
\answer{}
\end{enumerate}
	\item \textbf{Flexibility.} How can you switch between the solvers for the initial value problem? (max. 3 lines)
	\answer{}
	\item \textbf{Verification.} How did you verify the different parts of your implementation while coding? What tools did you use to debug your code? (max. 6 lines)
	\answer{}
	\item What value did you use for $\epsilon$ in the finite difference approximation? Why?
	\answer{}
	\item Which modifications would you need to make to your code to add another optimization algorithm? (max. 5 lines)
	\answer{}
	\item \textbf{Optional:} Mention the difficulties you encountered while implementing this part. (max.
10 lines)
	\answer{}
\end{enumerate}
\subsection*{Verify your code}
Do not forget to also include your figures separately in your ZIP!!!! (remove this line in your
submission)
\begin{figure}[!h]
	\centering
	\begin{subfigure}{0.49\linewidth}
		\centering
		%\includegraphics[width=\textwidth]{}
	\end{subfigure}
	\begin{subfigure}{0.49\linewidth}
		\centering
		%\includegraphics[width=\textwidth]{}
	\end{subfigure}
\end{figure}
\section*{Extra questions}
\begin{enumerate}
	\item What C++-features did you use to achieve this?
	\answer{}
	\item Did you make other improvements? How big is their effect on the execution time, the number of calls to the IVP solver or the memory usage?
	\answer{}
\end{enumerate}

\end{document}